\chapter {Recommendations}

There are quite a few improvements that can be made to the system that were mainly not implemented due to time constraints and/or unexpected problems.

LED light and photodiode insulation is a topic that was not addressed in this system. To increase the SNR of the system it is advisable to block outside sources of light from interfering with the measurements or light sources. It is advisable to use some form of black insulation to absorb the outside light rather than reflect it to reduce scattering and increase SNR. 

One way of doing this is to apply black epoxy around the LEDs and photodiodes sufficiently to block light leakage and external light. However, sourcing medical grade black epoxy is not an easy task; it is hard enough finding normal black epoxy. It is possible to add a pigment to white epoxy, but this is no easy task either. An alternative is to use some kind of black foam or mesh.

The design of the sensor is not ideal. If it were to be revised a double-sided PCB would be used or the sensor would be made larger to reduce the number of jumper wires. This would help to make the sensor more robust and make building the sensor easier. Additionally the traces would be made larger to reduce the chance of breakage when flexed. 

The LEDs do not rest against the patient's skin when the sensor is placed against the patient. This reduced the SNR of the system by allowing light in from outside sources and decreases the light intensity before reaching the area of interest. This was the unattended side-affect of using SMD LEDs with the OPT101 monolithic photodiodes. Perhaps the use of standard SMD photodiodes or non-SMD LEDs would fix this. However, the use of standard SMD photodiodes would require the addition of a transimpedance amplifier. Finding non-SMD LEDs would be hard task, the SMD LEDs were selected in this project because that was all that could be acquired for a reasonable price. 

Attaching the sensors to the patients head was not addressed in this project. This is more or less a luxury. For the testing phase it is somewhat sufficient to hold the sensors to the head by hand. This will not be suitable for the final product though; any movement from the area of interest will add noise into the measurements. Thus, current ideas for achieving this would be using something like a headband or hairband to hold the sensor down; something that would provide enough pressure to ensure the sensor doesn't move while still allowing comfort for the patient.

Currently, this microcontroller uses BJTs to control the infrared LEDS. Should a second revision be done, an LED driver which controls the LEDs by PWM (pulse width modulation) would be used rather than simple BJTs. This will increase the efficiency of the system and allow greater control over the LEDs. This would allow for more localized measurements and power savings. Being able to control the brightness of the LEDs can be very advantageous for a versatile system. It does, however, come at the cost of complexity and thus time.

The overall system would greatly benefit if the system could display the data in real time. This is not very hard to do, just time consuming. Though having this would be of great benefit in a clinical environment where seeing the results the as fast possible may save life. 

Perhaps the greatest downfall of this project was trying to accomplish too much in a short period of time. The author took on the task of designing everything from scratch. This may have been detrimental to the overall result of the project. Rather than spending time on correctly displaying the results or ensuring the system work correctly all the time, time was spent make the microcontroller. It probably would have been beneficial to the project to have used an existing embedded system and focus more on the overall result. However, the knowledge gained has been invaluable to the author and the experience obtained designing the microcontroller will more likely be more beneficial to the author in the long run.

\chapter {Conclusion}

Despite these limitations, a fairly low-cost, wireless portable Near Infrared Spectrophotometry system was successfully built that is able to measure haemodynamic signals. The results shown by the NIRS device created in this project are promising. It was able to measure brain activity in the brain not only during motor activity, but during intense thinking and visual stimulation. The results are fairly promising as well, there is a strong correlation between brain activity and the measurements taken. In addition, it is also able to directly measure muscle activity, which was an unexpected bonus. 

While NIRS devices are currently still in the research phase with very few commercial systems, it is hard to see it ever successfully replacing EEG or fMRI measurements. EEG systems have a very large marketshare with commercially available system everywhere for very cheap and have a slightly higher temporal resolution than NIRS. fMRI has a much higher spatial resolution than NIRS, at the cost of temporal resolution, but there are many times researchers or doctors need to see brain activity deep into the brain. Laser NIRS systems can penetrate deeper than LED based ones, but at an increased cost, complexity, and loss of portability. However, there is that niche market where EEG does not cut it due to electromagnetic interference or a device with spatial resolution is required but fMRI is deemed to costly, is unable to be used due to long waiting lists, or in the case the patient cannot be transferred to the MRI machine. For this niche market, NIRS is their saviour. 
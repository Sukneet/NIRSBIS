\chapter{Introduction}
\section{Background}
Near-infrared spectrophotometry (NIRS) is a non-invasive imaging technique that can be used to measure brain function. It provides a higher spatial resolution than EEG or MEG without subjecting the patient to radiation in PET or having the low temporal resolution in fMRI scans \cite{wolf08}. NIRS systems are also considerably more portable than PET or fMRI scanners without the messy string of wires involved with EEG or MEG. As such, they can be used on bed-ridden patients where transportation is not possible. NIRS systems can also be used for long periods and are not directly susceptible to electro-magnetic interference  \cite{wolf08}.  \cite{wolf08} provides a good discussion about the principles, applications, and technologies of NIRS. 

A NIRS system for measuring brain activity is of particular interest in neonatal brain activity measurement where fMRI is not recommended for use \cite{wolf08}. Such systems allow for the quantification of the haemodynamic response of the cortex to visual, tactile, or auditory stimuli \cite{mini08}.  

NIRS works on the principle that light passing through a tissue is variably attenuated depending on the constitution of the tissue \cite{wolf08}. A near-infrared light source (within the spectra 700nm - 900nm) is placed onto the tissue. The re-emergent light is then picked up by a detector and the light intensity is measured. Varying levels in oxyhaemoglobin (O$_{2}$Hb) and  deoxyhaemoglobin (HHb) cause the optical properties of the tissue to vary; the absorption and scattering of light passing though the tissue changes. NIRS is very sensitive to these changes and is, thus, able to measure the haemoglobin concentration in the order of per mills \cite{wolf05}.

NIRS is able to pick up fast neuronal signals and slow haemodynamic signals. It is currently the only available imaging technique that is able to detect both signals, which is its main advantage. Fast neuronal signals are changes directly associated with neuron activity. The response typically arrives within 200ms. Slow haemodynamic signals are changes in haemoglobin concentrations caused by an increase in local oxygen use and subsequently blood flow due to brain activity \cite{mini08}. Fast neuronal signals are typically measured by EEGs, while slow haemodynamic signals are typically measured by fMRI.

A Near-Infrared spectrophotometry system was selected as this paper's subject because it combines three of the author's interests; imaging, optics, and the brain. Additionally, the project could be done fairly cost effectively, costing roughly \$200-300.

\section{Definition}
The main scope of this project is to measure brain activity in a patient. In order to do this, near infrared spectrophotometry will be used, wherein near-infrared light is shown into the patient's head to measure changes of oxygenation that directly relate to brain activity.

The objective of this project is to design a near infrared spectrophotometry (NIRS) system for the measurement of brain activity in localized areas in the brain. A prototype of the design will be built and wirelessly interfaced with a computer. The time-varying haemodynamic signals will then be plotted on the computer versus time. 

It is important to note that the results will be qualitative rather than quantitative as a device that correctly measures the optical properties of the patients skin, skull, fat, etc is not in the scope of this project. Since these optical properties vary based on skin pigmentation, age, gender, and many other factors, such a device is needed to obtain exact quantitative measurements. However, without this device a NIRS system can still measure varying haemodynamic signals. The absolute haemodynamic signal values will be irrelevant, the change in the haemodynamic signals over time are the relevant measurements. 

Since a major target of NIRS systems are neonatal infants, it is critical to minimize distress by performing the measurements as comfortably as possible due to the frailty of newborns \cite{mini08}. This is achieved by not having wires attached from the system to a computer, so as to not confine movement. Additionally, since neonatal infants are small, the system should also be small and light to restrict the infant as little as possible. As such, the device should be portable, light, small, and wireless. 

Additionally, a portable, light, small, wireless device allows the device to used on freely moving subjects. This allows the device to measure brain activity in athletes during exercise, subjects in social environments, and in animals.  

As with most projects, especially at the undergraduate level, it is important to keep costs down. Since NIRS systems for brain imaging is a fairly new field with no currently available commercial systems, low-cost devices will help to bring NIRS devices to the general population. In order to compete with current EEG systems, the device has to be fairly cheap and affordable. Therefore, while many early NIRS systems required the use of lasers, this project will focus on using low-cost LEDs.

Safety is always a concern in any device, specifically in medical equipment. The use of LEDs help to make NIRS systems suitable for clinical environments. The use of Lasers in early systems brought in safety concerns pertaining to damage the eye.

Thus, this project will focus on building a prototype of a low-cost wireless, portable NIRS system that is able to measure the change in haemodynamic signals.

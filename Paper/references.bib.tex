

@article{wolf05,
	author={D. Haensse and P. Szabo and D. Brown and J. C. Fauchere and P. Niederer and H. U. Bucher and M. Wolf},
	year={2005},
	month={Jun 13},
	title={A new multichannel near infrared spectrophotometry system for functional studies of the brain in adults and neonates},
	journal={Optics express},
	volume={13},
	number={12},
	pages={4525-4538},
	note={JID: 101137103; ppublish},
	abstract={We have designed a versatile, multi-channel near-infrared spectrophotometry (NIRS) instrument for the purpose of mapping neuronal activation in the neonatal and adult brain in response to motor, tactile, and visual stimulation. The optical linearity, stability, and high signal to noise ratio (>70 dB) of the instrument were demonstrated using an in vitro validation procedure. In vivo measurements on the adult forearm were also performed. Changes in oxygenation, induced by arterial occlusion of the forearm, were recorded and were shown to compare well with measurements acquired using a conventional NIRS instrument. To demonstrate the capabilities of the instrument, functional measurements in adults and neonates were performed. The instrument exhibited the capability to differentiate with a spatial resolution in the order of cm, local activation patterns associated with a finger tapping sequence.},
	isbn={1094-4087; 1094-4087},
	language={eng}
}

@article{mini08,
	author={T. Muehlemann and D. Haensse and M. Wolf},
	year={2008},
	month={Jul 7},
	title={Wireless miniaturized in-vivo near infrared imaging},
	journal={Optics express},
	volume={16},
	number={14},
	pages={10323-10330},
	note={LR: 20081121; JID: 101137103; 7782-44-7 (Oxygen); ppublish},
	abstract={Our group measures tissue oxygenation and the cortical hemodynamic response to sensory stimuli applying continuous wave near-infrared imaging (NIRI). To improve the method's quality and applicability and to explore new fields in clinical practice and research, we developed a miniaturized wireless NIRI system. It was validated by measuring muscle oxygenation in a blood-flow occlusion experiment and brain activity in adults.},
	keywords={Amplifiers, Electronic; Biomedical Engineering/instrumentation/methods; Blood Flow Velocity; Diagnostic Imaging/methods; Equipment Design; Humans; Light; Male; Miniaturization; Muscles/metabolism/pathology; Optics and Photonics; Oxygen/metabolism; Regional Blood Flow/physiology; Software; Spectroscopy, Near-Infrared/instrumentation/methods},
	isbn={1094-4087; 1094-4087},
	language={eng}
}

@article{toro01,
	author={V. Toronov and A. Webb and J. H. Choi and M. Wolf and A. Michalos and E. Gratton and D. Hueber},
	year={2001},
	title={Investigation of human brain hemodynamics by simultaneous near-infrared spectroscopy and functional magnetic resonance imaging.},
	journal={Medical physics},
	volume={28},
	number={4},
	pages={521},
	isbn={0094-2405}
}

@article{wolf99,
	author={M. Wolf and M. Keel and V. Dietz and K. von Siebenthal and H. U. Bucher and O. Baenziger},
	year={1999},
	title={The influence of a clear layer on near-infrared spectrophotometry measurements using a liquid neonatal head phantom.},
	journal={Physics in medicine biology},
	volume={44},
	number={7},
	pages={1743},
	isbn={0031-9155}
}

@article{wolf08,
	author={M. Wolf and G. Morren and D. Haensse and T. Karen and U. Wolf and J. Fauchère and H. Bucher},
	year={2008},
	month={12/01},
	title={Near infrared spectroscopy to study the brain: an overview},
	journal={Opto-Electronics Review},
	volume={16},
	number={4},
	pages={413-419},
	note={M3: 10.2478/s11772-008-0042-z},
	abstract={Abstract This paper gives an overview of principles, technologies, and applications using near infrared spectrometry and imaging (NIRS and NIRI) to study brain function. The physical background is reviewed and technologies and their properties are discussed. Advantages and limitations of NIRI are described. The basic functional signals obtained by NIRI, the neuronal and the hemodynamic signal are described and in particular publications about the former are reviewed. Applications in adults and neonates are reviewed, too.},
	url={http://dx.doi.org/10.2478/s11772-008-0042-z}
}

@article{wray88,
	author={S. Wray and M. Cope and D. T. Delpy and J. S. Wyatt and E. O. Reynolds},
	year={1988},
	title={Characterization of the near infrared absorption spectra of cytochrome aa3 and haemoglobin for the non-invasive monitoring of cerebral oxygenation.},
	journal={Biochimica et biophysica acta},
	volume={933},
	number={1},
	pages={184},
	isbn={0006-3002}
}

@article{yama01,
	author={Y. Yamashita and A. Maki and H. Koizumi},
	year={2001},
	month={Jun},
	title={Wavelength dependence of the precision of noninvasive optical measurement of oxy-, deoxy-, and total-hemoglobin concentration},
	journal={Medical physics},
	volume={28},
	number={6},
	pages={1108-1114},
	note={LR: 20081121; JID: 0425746; 0 (Hemoglobins); 0 (Oxyhemoglobins); 9008-02-0 (deoxyhemoglobin); ppublish},
	abstract={The precision of noninvasive optical measurement of the concentration changes in oxy-, deoxy-, and total-hemoglobin depends on wavelength. For estimating the precision, we calculated the noise level of the concentration changes as the uncertainty in measurements using several wavelength pairs of light. Seven laser diodes (664-848 nm) were used simultaneously for spectroscopic measurement of brain activity during finger motor stimulation. We also used the analysis of error propagation from the uncertainty in direct measurements of absorbance changes to estimate indirectly the uncertainty of concentration changes. The measurement of the concentration changes made using an 830/664-nm pair are two times (oxy-Hb) and six times (deoxy-Hb) more precise than those made using an 830/782-nm pair.},
	keywords={Biophysical Phenomena; Biophysics; Brain/metabolism; Hemoglobins/analysis; Humans; Models, Biological; Optics and Photonics; Oxyhemoglobins/analysis; Spectroscopy, Near-Infrared/methods/statistics & numerical data},
	isbn={0094-2405; 0094-2405},
	language={eng}
}

@article{rosen05,
	author={A. Bozkurt and A. Rosen and H. Rosen and B. Onaral},
	year={2005},
	month={Apr 29},
	title={A portable near infrared spectroscopy system for bedside monitoring of newborn brain},
	journal={Biomedical engineering online},
	volume={4},
	number={1},
	pages={29},
	note={LR: 20091118; JID: 101147518; 0 (Oxyhemoglobins); OID: NLM: PMC1112605; 2005/03/01 [received]; 2005/04/29 [accepted]; 2005/04/29 [aheadofprint]; epublish},
	abstract={BACKGROUND: Newborns with critical health conditions are monitored in neonatal intensive care units (NICU). In NICU, one of the most important problems that they face is the risk of brain injury. There is a need for continuous monitoring of newborn's brain function to prevent any potential brain injury. This type of monitoring should not interfere with intensive care of the newborn. Therefore, it should be non-invasive and portable. METHODS: In this paper, a low-cost, battery operated, dual wavelength, continuous wave near infrared spectroscopy system for continuous bedside hemodynamic monitoring of neonatal brain is presented. The system has been designed to optimize SNR by optimizing the wavelength-multiplexing parameters with special emphasis on safety issues concerning burn injuries. SNR improvement by utilizing the entire dynamic range has been satisfied with modifications in analog circuitry. RESULTS AND CONCLUSION: As a result, a shot-limited SNR of 67 dB has been achieved for 10 Hz temporal resolution. The system can operate more than 30 hours without recharging when an off-the-shelf 1850 mAh-7.2 V battery is used. Laboratory tests with optical phantoms and preliminary data recorded in NICU demonstrate the potential of the system as a reliable clinical tool to be employed in the bedside regional monitoring of newborn brain metabolism under intensive care.},
	keywords={Brain/blood supply/metabolism; Diagnosis, Computer-Assisted/instrumentation/methods; Equipment Design; Equipment Failure Analysis; Humans; Infant, Newborn; Intensive Care, Neonatal/methods; Male; Monitoring, Physiologic/instrumentation; Oximetry/instrumentation/methods; Oxyhemoglobins/analysis; Point-of-Care Systems; Spectrophotometry, Infrared/instrumentation/methods},
	isbn={1475-925X; 1475-925X},
	language={eng}
}

@article{chance97,
	author={A. Villringer and B. Chance},
	year={1997},
	month={Oct},
	title={Non-invasive optical spectroscopy and imaging of human brain function},
	journal={Trends in neurosciences},
	volume={20},
	number={10},
	pages={435-442},
	note={LR: 20061115; JID: 7808616; RF: 90; ppublish},
	abstract={Brain activity is associated with changes in optical properties of brain tissue. Optical measurements during brain activation can assess haemoglobin oxygenation, cytochrome-c-oxidase redox state, and two types of changes in light scattering reflecting either membrane potential (fast signal) or cell swelling (slow signal), respectively. In previous studies of exposed brain tissue, optical imaging of brain activity has been achieved at high temporal and microscopical spatial resolution. Now, using near-infrared light that can penetrate biological tissue reasonably well, it has become possible to assess brain activity in human subjects through the intact skull non-invasively. After early studies employing single-site near-infrared spectroscopy, first near-infrared imaging devices are being applied successfully for low-resolution functional brain imaging. Advantages of the optical methods include biochemical specificity, a temporal resolution in the millisecond range, the potential of measuring intracellular and intravascular events simultaneously and the portability of the devices enabling bedside examinations.},
	keywords={Brain/anatomy & histology/enzymology/physiology; Brain Chemistry/physiology; Humans; Spectroscopy, Near-Infrared},
	isbn={0166-2236; 0166-2236},
	language={eng}
}

@article{yam02,
	author={H. Zhao and Y. Tanikawa and F. Gao and Y. Onodera and A. Sassaroli and K. Tanaka and Y. Yamada},
	year={2002},
	month={Jun 21},
	title={Maps of optical differential pathlength factor of human adult forehead, somatosensory motor and occipital regions at multi-wavelengths in NIR},
	journal={Physics in Medicine and Biology},
	volume={47},
	number={12},
	pages={2075-2093},
	note={LR: 20061115; JID: 0401220; ppublish},
	abstract={The optical differential pathlength factor (DPF) is an important parameter for physiological measurement using near infrared spectroscopy, but for the human adult head it has been available only for the forehead. Here we report measured DPF results for the forehead, somatosensory motor and occipital regions from measurements on 11 adult volunteers using a time-resolved optical imaging system. The optode separation was about 30 mm and the wavelengths used were 759 nm, 799 nm and 834 nm. Measured DPFs were 7.25 for the central forehead and 6.25 for the temple region at 799 nm. For the central somatosensory and occipital areas (10 mm above the inion), DPFs at 799 nm are 7.5 and 8.75, respectively. Less than 10\% decreases of DPF for all these regions were observed when the wavelength increased from 759 nm to 834 nm. To compare these DPF maps with the anatomical structure of the head, a Monte Carlo simulation was carried out to calculate DPF for these regions by using a two-layered semi-infinite model and assuming the thickness of the upper layer to be the sum of the thicknesses of scalp and skull, which was measured from MRI images of a subject's head. The DPF data will be useful for quantitative monitoring of the haemodynamic changes occurring in adult heads.},
	keywords={Brain/pathology; Forehead/pathology; Humans; Light; Magnetic Resonance Imaging; Models, Statistical; Monte Carlo Method; Spectroscopy, Near-Infrared/methods; Time Factors},
	isbn={0031-9155; 0031-9155},
	language={eng}
}
@article{eke06,
	author={L. Kocsis and P. Herman and A. Eke},
	year={2006},
	month={Mar 7},
	title={The modified Beer-Lambert law revisited},
	journal={Physics in Medicine and Biology},
	volume={51},
	number={5},
	pages={N91-8},
	note={LR: 20061115; JID: 0401220; 0 (Hemoglobins); 0 (Oxyhemoglobins); 7732-18-5 (Water); 9008-02-0 (deoxyhemoglobin); 2006/02/15 [aheadofprint]; ppublish},
	abstract={The modified Beer-Lambert law (MBLL) is the basis of continuous-wave near-infrared tissue spectroscopy (cwNIRS). The differential form of MBLL (dMBLL) states that the change in light attenuation is proportional to the changes in the concentrations of tissue chromophores, mainly oxy- and deoxyhaemoglobin. If attenuation changes are measured at two or more wavelengths, concentration changes can be calculated. The dMBLL is based on two assumptions: (1) the absorption of the tissue changes homogeneously, and (2) the scattering loss is constant. It is known that absorption changes are usually inhomogeneous, and therefore dMBLL underestimates the changes in concentrations (partial volume effect) and every calculated value is influenced by the change in the concentration of other chromophores (cross-talk between chromophores). However, the error introduced by the second assumption (cross-talk of scattering changes) has not been assessed previously. An analytically treatable special case (semi-infinite, homogeneous medium, with optical properties of the cerebral cortex) is utilized here to estimate its order of magnitude. We show that the per cent change of the transport scattering coefficient and that of the absorption coefficient have an approximately equal effect on the changes of attenuation, and a 1\% increase in scattering increases the estimated concentration changes by about 0.5 microM.},
	keywords={Absorption; Hemoglobins/chemistry; Models, Theoretical; Oxyhemoglobins/chemistry; Scattering, Radiation; Spectroscopy, Near-Infrared; Water/chemistry},
	isbn={0031-9155; 0031-9155},
	language={eng}
}


@article{freeman03,
	author={Jeffrey K. Thompson and Matthew R. Peterson and Ralph D. Freeman},
	year={2003},
	month={February 14, 2003},
	title={Single-Neuron Activity and Tissue Oxygenation in the Cerebral Cortex},
	journal={Science},
	volume={299},
	number={5609},
	pages={1070-1072},
	url={http://www.sciencemag.org/cgi/content/abstract/299/5609/1070}
}

